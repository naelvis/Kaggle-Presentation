%%%%%%%%%%%%%%%%%%%%%%%%%%%%%%%%%%%%%%%%%%%%%%%%%%%%%%%%%%%%
%%  This Beamer template was created by Cameron Bracken.
%%  Anyone can freely use or modify it for any purpose
%%  without attribution.
%%
%%  Last Modified: January 9, 2009
%%

\documentclass[xcolor=x11names, compress]{beamer}

%% General document %%%%%%%%%%%%%%%%%%%%%%%%%%%%%%%%%%
\usepackage{graphicx}
\usepackage{tikz}
\usetikzlibrary{decorations.fractals}
%%%%%%%%%%%%%%%%%%%%%%%%%%%%%%%%%%%%%%%%%%%%%%%%%%%%%%

\usepackage{amsmath,amsthm,amssymb,amsfonts, mathrsfs, textcomp}

%% Beamer Layout %%%%%%%%%%%%%%%%%%%%%%%%%%%%%%%%%%
\useoutertheme[subsection=false,shadow]{miniframes}
\useinnertheme{default}
\usefonttheme{serif}
\usepackage{palatino}

\setbeamerfont{title like}{shape=\scshape}
\setbeamerfont{frametitle}{shape=\scshape}

\setbeamercolor*{lower separation line head}{bg=DeepSkyBlue4} 
\setbeamercolor*{normal text}{fg=black,bg=white} 
\setbeamercolor*{alerted text}{fg=red} 
\setbeamercolor*{example text}{fg=black} 
\setbeamercolor*{structure}{fg=black} 
 
\setbeamercolor*{palette tertiary}{fg=black,bg=black!10} 
\setbeamercolor*{palette quaternary}{fg=black,bg=black!10} 

\renewcommand{\(}{\begin{columns}}
\renewcommand{\)}{\end{columns}}
\newcommand{\<}[1]{\begin{column}{#1}}
\renewcommand{\>}{\end{column}}
%%%%%%%%%%%%%%%%%%%%%%%%%%%%%%%%%%%%%%%%%%%%%%%%%%

\theoremstyle{definition}
\newtheorem{definizione}{Definizione}
\theoremstyle{plain}
\newtheorem{teorema}{Teorema}


\begin{document}


%%%%%%%%%%%%%%%%%%%%%%%%%%%%%%%%%%%%%%%%%%%%%%%%%%%%%%
%%%%%%%%%%%%%%%%%%%%%%%%%%%%%%%%%%%%%%%%%%%%%%%%%%%%%%
\begin{frame}
\title{Ehrenfest Chains \\and Poincar\'e's Recurrence Theorem}
%\subtitle{SUBTITLE}
\author{
	Nelvis Fornasin\\
	{\it University of Trento}\\
}
\date{
	\includegraphics[scale=0.20]{logoUNITN}
	\\
	\vspace{1cm}
	23 July 2014
}
\titlepage
\end{frame}

%%%%%%%%%%%%%%%%%%%%%%%%%%%%%%%%%%%%%%%%%%%%%%%%%%%%%%
%%%%%%%%%%%%%%%%%%%%%%%%%%%%%%%%%%%%%%%%%%%%%%%%%%%%%%
\begin{frame}
\tableofcontents
\end{frame}

%%%%%%%%%%%%%%%%%%%%%%%%%%%%%%%%%%%%%%%%%%%%%%%%%%%%%%
%%%%%%%%%%%%%%%%%%%%%%%%%%%%%%%%%%%%%%%%%%%%%%%%%%%%%%
\section{\scshape Introduction}
\subsection{A matter of consistency}
\begin{frame}{A matter of consistency}
\large{Consistency between the $2^{nd}$ Principle of Thermodynamics and microscopical models:\\}
\begin{enumerate}
\item<1->$2^{nd}$ Principle $\nleftrightarrow$ Classical Mechanics;
\item<2->$2^{nd}$ Principle $\overbrace{\longleftrightarrow}^{?}$ Stochastic model;
\item<2->$2^{nd}$ Principle $\overbrace{\longleftrightarrow}^{?}$ Quantum Mechanics;
\end{enumerate}
\onslide<2>{$\quad$\newline We start by considering the stochastic model.}
\end{frame}

%%%%%%%%%%%%%%%%%%%%%%%%%%%%%%%%%%%%%%%%%%%%%%%%%%%%%%
%%%%%%%%%%%%%%%%%%%%%%%%%%%%%%%%%%%%%%%%%%%%%%%%%%%%%%
\section{\scshape Ehrenfest Chains}
\subsection{The model}
\begin{frame}{The model}
\includegraphics[scale=0.2]{ehchain1}
\end{frame}

%%%%%%%%%%%%%%%%%%%%%%%%%%%%%%%%%%%%%%%%%%%%%%%%%%%%%%
%%%%%%%%%%%%%%%%%%%%%%%%%%%%%%%%%%%%%%%%%%%%%%%%%%%%%%
\begin{frame}{The model}
\large{Features of Ehrenfest Chains:\\}
\begin{itemize}
\item Purely stochastic;
\item Time is a discrete variable;
\item Markov property.
\end{itemize}
$\quad$\newline We can analyze the model by means of the theory of Markov Chains!
\end{frame}

%%%%%%%%%%%%%%%%%%%%%%%%%%%%%%%%%%%%%%%%%%%%%%%%%%%%%%
%%%%%%%%%%%%%%%%%%%%%%%%%%%%%%%%%%%%%%%%%%%%%%%%%%%%%%
\subsection{Markov Chains}
\begin{frame}{Markov Chains}
\begin{definition}[Markov chain]
	A discrete stochastic process $X = \{X_n, n\geq 0\}$ on the state space $S$ is said to be a \emph{Markov chain} if for any sequence $\{x_i\}_i\subset S$
	\[
	P\{X_{n+1}=x_{n+1}\ |\ X_k = x_k,\ 0 \leq k \leq n\} = P\{X_{n+1}=x_{n+1}\ |\ X_n = x_n\}
	\]
\end{definition}
\end{frame}

%%%%%%%%%%%%%%%%%%%%%%%%%%%%%%%%%%%%%%%%%%%%%%%%%%%%%%
%%%%%%%%%%%%%%%%%%%%%%%%%%%%%%%%%%%%%%%%%%%%%%%%%%%%%%
\begin{frame}{Markov Chains}
\begin{definition}[Communication between states]
We say that a state $x$ \emph{leads} to state $y$ and write $x\longrightarrow y$ if $\exists n\geq 0\ :\ p^n(x,y) > 0$. If both $x\longrightarrow y$ and $x\longleftarrow y$ hold, we say that $x$ and $y$ \emph{communicate}, and write $x\longleftrightarrow y$.
\end{definition}
$\quad$\newline
The relation $\longleftrightarrow$ is an equivalence relation!
$\quad$\newline
{\begin{definition}[Irreducibility]
An MC is said to be irreducible if its state space is an equivalence class with respect to $\longleftrightarrow$.
\end{definition}}
\end{frame}

%%%%%%%%%%%%%%%%%%%%%%%%%%%%%%%%%%%%%%%%%%%%%%%%%%%%%%
%%%%%%%%%%%%%%%%%%%%%%%%%%%%%%%%%%%%%%%%%%%%%%%%%%%%%%
\begin{frame}{Markov Chains}
The following, fundamental result holds:
\begin{theorem}\label{thm2}
Let $\{X_n\}_n$ be an irreducible MC with finite state space S: then all the states of $\{X_n\}_n$ are recurrent, and the MC is said to be recurrent.
\end{theorem}
\onslide<2>{Furthermore, for the EC it is possible to compute a mean recurrence time as
\[
\tau(x) = 2^{2N}\frac{x!(2N-x)!}{2N!}
\]
For a state with $x$ particles out of all the $2N$ particles involved.}

\end{frame}

%%%%%%%%%%%%%%%%%%%%%%%%%%%%%%%%%%%%%%%%%%%%%%%%%%%%%%
%%%%%%%%%%%%%%%%%%%%%%%%%%%%%%%%%%%%%%%%%%%%%%%%%%%%%%
\section{\scshape Poincar\'e's Recurrence Theorem}
\begin{frame}{A matter of consistency}
\large{Consistency between the $2^{nd}$ Principle of Thermodynamics and microscopical models:\\}
\begin{enumerate}
\item$2^{nd}$ Principle $\nleftrightarrow$ Classical Mechanics;
\item$2^{nd}$ Principle $\nleftrightarrow$ Stochastic model; (!)
\item$2^{nd}$ Principle $\overbrace{\longleftrightarrow}^{?}$ Quantum Mechanics;
\end{enumerate}
$\quad$\newline What about QM?
\end{frame}

%%%%%%%%%%%%%%%%%%%%%%%%%%%%%%%%%%%%%%%%%%%%%%%%%%%%%%
%%%%%%%%%%%%%%%%%%%%%%%%%%%%%%%%%%%%%%%%%%%%%%%%%%%%%%
\subsection{Physical hypotheses}
\begin{frame}{Physical hypotheses}
\large{Features of the system:}
\begin{itemize}
\item $N$ particles with mass $m$;
\item No mutual interactions;
\item Enclosure in a finite volume;
\item No spin.
\end{itemize}
$\quad$\newline Main problem: exchange degeneracy.
\end{frame}

%%%%%%%%%%%%%%%%%%%%%%%%%%%%%%%%%%%%%%%%%%%%%%%%%%%%%%
%%%%%%%%%%%%%%%%%%%%%%%%%%%%%%%%%%%%%%%%%%%%%%%%%%%%%%
\subsection{Mathematical issues}
\begin{frame}{Mathematical issues}
\large{Main points we need to discuss:}
\begin{enumerate}
\item How do we swap functions?
\item What is a symmetrization?
\end{enumerate}
$\quad$\newline And then we can prove the theorem!
\end{frame}

%%%%%%%%%%%%%%%%%%%%%%%%%%%%%%%%%%%%%%%%%%%%%%%%%%%%%%
%%%%%%%%%%%%%%%%%%%%%%%%%%%%%%%%%%%%%%%%%%%%%%%%%%%%%%
\begin{frame}{Mathematical issues}
\framesubtitle{1. Swapping functions}
We work with:
\begin{enumerate}
\item The group of N-permutations: $\mathcal P_N$;
\item The Hilbertian tensor product of N copies of the single particle's Hilbert space: $\mathcal H^{\otimes N}$.
\end{enumerate}
{$\quad$\newline $\Rightarrow$ There exists a representation of $\mathcal P_N$ on $\mathcal H^{\otimes N}!$
}
\end{frame}

%%%%%%%%%%%%%%%%%%%%%%%%%%%%%%%%%%%%%%%%%%%%%%%%%%%%%%
%%%%%%%%%%%%%%%%%%%%%%%%%%%%%%%%%%%%%%%%%%%%%%%%%%%%%%
\begin{frame}{Mathematical issues}
\framesubtitle{2. Symmetrizing}
Symmetrization and anti-symmetrization are indeed projections:
\begin{itemize}
\item $\mathcal A: \mathcal H^{\otimes N}\to\Lambda(\mathcal H^{\otimes N}) : \phi\mapsto\frac{1}{N!}\sum_{\sigma\in\mathscr P_N}\epsilon_\sigma\sigma^\otimes\phi$
\item $\mathcal S: \mathcal H^{\otimes N}\to S(\mathcal H^{\otimes N}) : \phi\mapsto\frac{1}{N!}\sum_{\sigma\in\mathscr P_N}\sigma^\otimes\phi$
\end{itemize}
$\quad$\newline Where $\Lambda(\mathcal H^{\otimes N})$ and $S(\mathcal H^{\otimes N})$ are respectively the set of all anti-symmetric and symmetric tensors of $\mathcal H^{\otimes N}$.
\end{frame}

%%%%%%%%%%%%%%%%%%%%%%%%%%%%%%%%%%%%%%%%%%%%%%%%%%%%%%
%%%%%%%%%%%%%%%%%%%%%%%%%%%%%%%%%%%%%%%%%%%%%%%%%%%%%%
\subsection{The theorem}
\begin{frame}{The theorem}
\large{
\begin{theorem}[Poincar\'e's Recurrence Theorem]
Let $\psi(t, x)$ the wavefunction of the considered system.\newline
Then $\forall t_0\in\mathbb R, \forall \varepsilon>0\ \exists\ T=T(\varepsilon)>t_0$ such that
\[
||\psi(t_0,\cdot)-\psi(T,\cdot)||<\varepsilon
\]
\end{theorem}
}
Central point:
\[
\forall \varepsilon > 0,\ \forall x\in\mathbb R,\ \exists n,m\in\mathbb N\ :\ |nx-m|<\varepsilon
\]
\end{frame}

%%%%%%%%%%%%%%%%%%%%%%%%%%%%%%%%%%%%%%%%%%%%%%%%%%%%%%
%%%%%%%%%%%%%%%%%%%%%%%%%%%%%%%%%%%%%%%%%%%%%%%%%%%%%%
\section{\scshape Conclusion}
\subsection{A paradox?}
\begin{frame}{A paradox?}
\begin{itemize}
\item Difference between \emph{figures} and \emph{real objects}: we should look for contradiction between theoretical previsions and experimental results;
\item Poincar\'e's theorem and Ehrenfest's model highlight the fact we still haven't found a microscopical theory consistent with the $2^{nd}$ Law of Thermodynamics;
\item Implying such a theory exists...
\end{itemize}
\end{frame}

%%%%%%%%%%%%%%%%%%%%%%%%%%%%%%%%%%%%%%%%%%%%%%%%%%%%%%
%%%%%%%%%%%%%%%%%%%%%%%%%%%%%%%%%%%%%%%%%%%%%%%%%%%%%%
\section{ }
\begin{frame}
\end{frame}

\end{document}